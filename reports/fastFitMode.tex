\title{Fast-fit-mode: partial results}
\author{Robert Poenaru
}
\date{\today}
\documentclass[11pt]{article}
%\documentclass{acmconf}

\usepackage[paper=a4paper,dvips,top=1.5cm,left=1.5cm,right=1.5cm,
    foot=1cm,bottom=1.5cm]{geometry}

\usepackage{times}
\usepackage{graphicx}
\usepackage[fleqn]{amsmath}
\usepackage{amsfonts}
\usepackage{amssymb}
\usepackage{amsthm}
\usepackage{amsopn}
\usepackage{xspace}
\usepackage{array}
\usepackage{epsfig}
\usepackage{lipsum}
\usepackage{color}

\begin{document}

\maketitle
\section{Rezultate preliminarii}
Am modificat programul astfel incat la fiecare pas, in loc sa verific daca frecventa de wobbling $\omega_\theta$ (unde frecventa de wobbling $\omega=f(I)$ este o functie de spin), fac o verificare asupra perechii:$$\Omega=\left\{\omega_\theta,\omega^\text{chiral}_{\theta'}\right\}$$ sa fie formata doar din numere reale si pozitive.
Evident ca cea de a doua frecventa, notata de mine $\omega^\text{chiral}$, nu este altceva decat frecventa pentru unghiul $\theta'=\theta+\pi$.
\par \emph{In acest fel, am izolat complet problema spinilor mici in care frecventa "chirala" ar fi putut fi complexa.}
\section{Fast fit mode}
Am luat pasi destui de mari in cautarea minimului functiei RMS (pasi mari atat pentru momentele de inertie $\mathcal{I}_k$ cat si pentru unghiul $\theta$). Pentru pasi mari, o cautare are loc in cateva minute, de aceea am numit programul \emph{fast-fit-mode}.
\subsection{Rezultate numerice}
Am obtinut urmatorul set de parametrii $\mathbf{X}=\left\{\theta,\mathcal{I}_1,\mathcal{I}_2,\mathcal{I}_3\right\}$:

\begin{itemize}
    \item Unghiul de coupling: $\theta=-54$\hspace{1cm} ({\color{red} Rez. anterioare: $\theta=-71$}),
    \item Momentele de inertie: $\mathcal{I}_1=91$, $\mathcal{I}_2=11$, $\mathcal{I}_3=46$\hspace{1cm} ({\color{red} Rez. anterioare: $\mathcal{I}_1=89$, $\mathcal{I}_2=12$, $\mathcal{I}_3=48$}),
    \item $E_\text{RMS}=0.174611$\hspace{1cm} ({\color{red} Rez. anterioare: $E_\text{RMS}=0.174452$}).
\end{itemize}

\subsection{Concluzii}

\begin{itemize}
    \item Chiar si cu pasi mari, programul a gasit un RMS destul de apropiat de cel initial, desi este putin mai mare.
    \item Problema frecventelor de wobbling pentru $\theta'=\theta+\pi$ la spini mici $I\approx 5\hbar$ a fost rezolvata. Perechea $\Omega$ are acum doar numere reale pozitive.
    \item Concret pentru cazul nostru, in care initial aveam probleme la spinul $11/2$: $\Omega(I=11/2)=\left\{0.158867,0.0857974\right\}$
    \item \textbf{URMEAZA SA PUN LA RULAT PROGRAM CU PASI FOARTE MARUNTI. ESTE ASTEPTATA O IMBUNATATIRE A RMS-ULUI + FRECVENTE REALE LA SPINI MICI.}
\end{itemize}

\end{document}
